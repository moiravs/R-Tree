\documentclass[utf8]{article}
\usepackage[utf8]{inputenc}

\usepackage[parfill]{parskip}
\usepackage{booktabs}
\usepackage{amsmath}
\usepackage{amssymb}
\usepackage{amsfonts}
\usepackage{graphicx}
\usepackage{float}
\usepackage{listingsutf8}
\usepackage{listings}
\usepackage{graphicx}
\usepackage{hyperref}
\usepackage{fullpage}
\usepackage{lipsum}
\usepackage{multirow}
\setlength\parindent{24pt}
\begin{document}

\begin{titlepage}


    \author{Andrius Ezerskis \& Moïra Vanderslagmolen}
    \title{Projet d'Algorithmique: R-trees}
    \date{Avril 2023}
    \maketitle
\end{titlepage}
\tableofcontents
\newpage
\begin{large}


    \section{Introduction}
    \indent
    \par
    Ce rapport se divise en plusieurs parties. Tout d'abord, nous expliquons la
    structure de notre code en expliquant le rôle de chaque classe et leurs
    méthodes associées. Ensuite, nous parlerons des optimisations effectuées afin
    d'améliorer la vitesse de nos algorithmes. Enfin, nous analyserons les tests
    réalisés dans le cadre de ce projet et les influences des différentes
    optimisations sur le temps d'exécution.
    \par
    \section{Structure du code}

    \par
    \subsection{MBRNode}
    \indent
    \par
    MBRNode représente les noeuds des R-Tree. Elle contient un label, un Minimum Bouding Rectangle(MBR),un polygone,
    des enfants (sauf si c'est une feuille) et un parent (sauf si c'est la racine de
    l'arbre). Si le noeud n'est pas une feuille, alors son label sera `SplitSeed',
    et son MBR contiendra l'union de tous les MBR de ses enfants.
    \par

    \subsection{RTree}\label{RTree}
    \indent
    \par
    RTree est une classe abstraite. Elle permet de regrouper ensemble les méthodes
    communes à RTreeLinear et RTreeQuadratic, par exemple la recherche d'un noeud
    (searchNode), l'initialisation de la classe, les méthodes split, pickNext,
    chooseNode et addLeaf.
    \par

    \subsubsection{AddLeaf}\label{addLeaf}
    \indent
    \par
    Lorsque nous voulons ajouter un nouveau node, la méthode addLeaf() est appelée
    avec la racine de l'arbre en paramètre. Si la racine de l'arbre n'a pas
    d'enfants ou que son premier enfant n'a pas d'enfant, alors le nouveau node
    est rajouté à la racine de l'arbre. Ensuite, nous augmentons le MBR du parent,
    ainsi que celui de ses parents grâce à la méthode expandMBR(). Sinon, nous
    choisissons un noeud grâce à la méthode \nameref{chooseNode} et puis la méthode \nameref{addLeaf}
    est de nouveau appelée, avec cette fois-ci le node choisi par la méthode
    \nameref{chooseNode}.
    \par
    \indent
    Si le nombre d'enfants du node passé en paramètre dépasse un nombre N, alors
    un split est effectué.
    \par
    \subsubsection{Split}\label{split}
    \indent
    \par
    L'algorithme de split commence par appeler le pickSeeds quadratique ou
    linéaire afin d'obtenir deux seeds. \newline Ensuite, la
    méthode pickNext s'assure de choisir, pour chaque sous-noeud, la seed dont le
    MBR augmente le moins avec ce sous-noeud, et le sous-noeud devient alors
    l'enfant de cette seed. \newline Pour finir, les deux splitSeeds obtenus via
    pickSeeds reçoivent comme parent le node (celui que l'on doit split) et,
    réciproquement, le node reçoit les deux splitSeeds comme ses enfants. Le noeud
    a été divisé.
    \par

    \subsubsection{ChooseNode}\label{chooseNode}
    \indent
    \par
    La méthode chooseNode prend en paramètre le bestNode et le noeud à ajouter.
    Si le bestNode n'a pas d'enfants ou si son premier enfant n'a pas d'enfants,
    alors le bestNode est renvoyé. Sinon, nous itérons à travers les enfants de
    bestNode. Nous choissisons le noeud qui minimise l'aire de l'union de ce noeud avec le noeud à ajouter. Nous appelons ensuite récursivement la méthode chooseNode.

    \par
    \subsubsection{Search}
    \par
    \indent
    La fonction search commence à partir de la racine. Si la racine n'a pas
    d'enfants, si le MBR de la racine et si le polygone de la racine contiennent
    le point, alors la racine est retournée. Sinon, si la racine a des enfants et
    si son MBR contient le point cherché, alors nous itérons dans les enfants et
    appelons la méthode search sur chaque enfant.

    \par
    \subsection{RTreeLinear et RTreeQuadratic}\label{RTreeLinear}
    \par
    \indent

    RTreeLinear représente l'implémentation du R-Tree avec l'algorithme de split
    linéaire et RTreeQuadratic représente le R-Tree avec l'algorithme de split quadratique.
    \par

    \subsubsection{PickSeeds quadratique}\label{PickSeeds quadratique}
    \par
    \indent
    Le pickSeeds quadratique cherche les deux seeds qui maximisent l'aire de l'enveloppe qui les entoure.
    Pour ce faire, nous itérons à travers tous les sous-noeuds du noeud donné en paramètre,
    et nous calculons la différence d'aire entre l'union de deux sous-noeuds consécutifs et l'aire de
    ces derniers séparement.
    Le pickSeeds retourne donc un vecteur contenant les deux seeds choisies.
    \par
    \indent
    La complexité de cette méthode est $O(n²)$.
    \par
    \subsubsection{PickSeeds linéaire}\label{PickSeeds lineaire}
    \indent
    \par
    Le pickSeeds linéaire commence par itérer dans chaque sous-noeud du noeud
    que nous voulons split. Ensuite, nous choisissons le noeud dont le côté droit du MBR
    associé a le plus petit x, le noeud dont le côté gauche du MBR a le plus grand
    x. Nous faisons de même avec la dimension y. Par après, nous testons si les
    nodes sont tous les mêmes, ou si les noeuds d'une dimension sont les mêmes.
    Dans le premier cas, nous renvoyons null et le split ne se produit pas. En
    effet, si les seeds trouvées sont les mêmes, alors le split ne se produira pas
    correctement car tous les noeuds iront donc dans la première seed. Dans le
    deuxième cas, nous choisissons l'autre dimension dont les seeds ne sont pas
    les mêmes.
    \par
    \indent
    \par
    Si les noeuds sont tous différents, nous calculons la normalisation, qui
    correspond au ratio entre les côtés internes et les côtés externes pour chaque
    dimension (dans ce projet-ci, la dimension x et y). Nous prenons ensuite la
    normalisation la plus élevée.
    \par
    \par
    \indent
    La complexité de cette méthode est $O(n)$.
    \par

    \subsection{FileLoader}
    \indent
    \par
    Cette classe permet de charger le shapefile en mémoire. Nous avons implémenté
    cette classe afin de facilement changer la façon dont les fichiers sont chargés.


    \section{Optimisations}
    \subsection{Multi-polygones}
    \indent
    \par
    Notre première optimisation se base sur les multi-polygones. En effet, comme
    expliqué dans les consignes du projet d'algorithmique, nous avons remarqué que
    des polygones très étendu comme la france (voir \nameref{Figure 1})
    prenaient énormément de place. Nous avons donc décidé de séparer les
    multi-polygones en polygones (voir \nameref{Figure 2}). Cela nous a
    permis de gagner quelques secondes lors des tests dont nous parlerons au
    chapitre \nameref{Analyse des tests}.

    \subsection{PickSeeds quadratique}
    \indent
    \par
    Dans le pickSeeds quadratique, nous itérons à travers le vecteur d'enfants du node à
    split et nous prenons les deux noeuds les plus éloignés. Nous avons vite
    remarqué qu'il n'était pas nécéssaire de faire un double for en itérant chaque
    élément dans le vecteur. En effet, si nous calculons l'aire pour le noeud A
    et pour le noeud B, il ne faut pas recalculer pour le noeud B et le noeud A, de
    même qu'il ne faut pas calculer l'aire pour le noeud A et le noeud A. Nous
    faisons donc deux boucles for, l'une itérant dans le vecteur entier de
    sous-noeuds, l'autre commençant à l'indice de la précédente boucle + 1. De cette
    manière, nous commençons avec le noeud A et calculons avec le noeud B et tous
    les autres noeuds restants, puis le noeud B avec le noeud C et ainsi de suite. Cela nous permet de réduire le temps d'exécution.

    \subsection{Nombre d'enfants maximum}
    \indent
    \par
    Au début, nous avions N (le nombre d'enfants maximum pour les RTree) en
    attribut de la classe RTree. Afin d'optimiser notre programme, nous avons
    décidé de le passer en paramètre. Grâce à cette amélioration, les 8 tests que
    nous avons fait tourner ont pris 4,429 secondes au total au lieu de 5,865
    secondes.


    \section{Expérience sur données réelles}
    \subsection{Présentation des tests}
    \begin{enumerate}
        \item Carte de la belgique - Algorithme linéaire - Campus universitaire
        \item Carte de la belgique - Algorithme linéaire - Point pas dans le polygone
        \item Carte du monde - Algorithme linéaire - Kazakhstan
        \item Carte du monde - Algorithme linéaire - Canada
        \item Carte de la france - Algorithme linéaire - Auvergne
        \item Carte de la france - Algorithme linéaire - Guyane
        \item Carte de la belgique - Algorithme quadratique - Campus universitaire
        \item Carte de la belgique - Algorithme quadratique - Point pas dans le polygone
        \item Carte du monde - Algorithme quadratique - Kazakhstan
        \item Carte du monde - Algorithme quadratique - Canada
        \item Carte de la france - Algorithme quadratique - Auvergne
        \item Carte de la france - Algorithme quadratique - Guyane
        \item Carte du japon - Algorithme quadratique - Gifu
        \item Carte du japon - Algorithme quadratique - Gifu
    \end{enumerate}

    \subsection{Présentation des cartes}
    \begin{enumerate}
        \item Carte de la belgique: 19795 polygones
        \item Carte du monde: 251 polygones
        \item Carte de la france: 18 polygones
        \item Carte du japon: 1892 polygones
    \end{enumerate}

    \indent
    \par
    Ces tests ont été effectués sur une machine Lenovo Yoga 7. La distribution
    linux est KDE Neon 5.27 et la release ubuntu est la 22.04.
    \indent

    \subsection{Résultats des tests}
    \indent
    \par
    Temps en millisecondes de la fonction search en fonction des différents tests et valeurs.

    \begin{tabular}{ |p{3cm}||p{3cm}|p{3cm}|p{3cm}|  }
        \hline
        \multicolumn{4}{|c|}{Temps en millisecondes de la fonction search} \\
        \hline
        Test   & N=4 & N = 500 & N = 10000                                 \\
        \hline
        Test1  & 15  & 15      & 17                                        \\
        Test2  & 1   & 1       & 1                                         \\
        Test3  & 76  & 8       & 11                                        \\
        Test4  & 123 & 214     & 219                                       \\
        Test5  & 23  & 68      & -                                         \\
        Test6  & 3   & 3       & -                                         \\
        Test7  & 2   & 0       & 2                                         \\
        Test8  & 1   & 1       & 4                                         \\
        Test9  & 3   & 2       & 2                                         \\
        Test10 & 29  & 76      & 73                                        \\
        Test11 & 29  & 37      & -                                         \\
        Test12 & 2   & 2       & -                                         \\
        Test13 & 2   & 2       & -                                         \\
        Test14 & 2   & 2       & -                                         \\

        \hline
    \end{tabular}

    \begin{tabular}{ |p{3cm}||p{3cm}|p{3cm}|p{3cm}|  }
        \hline
        \multicolumn{4}{|c|}{Temps en millisecondes de la fonction search avec l'optimisation} \\
        \hline
        Test   & N=4 & N = 500 & N = 10000                                                     \\
        \hline
        Test1  & 12  & 14      & 19                                                            \\
        Test2  & 1   & 1       & 6                                                             \\
        Test3  & 11  & 12      & 9                                                             \\
        Test4  & 20  & 28      & 23                                                            \\
        Test5  & 99  & 95      & -                                                             \\
        Test6  & 3   & 2       & -                                                             \\
        Test7  & 3   & 0       & 2                                                             \\
        Test8  & 0   & 3       & 2                                                             \\
        Test9  & 1   & 2       & 2                                                             \\
        Test10 & 5   & 5       & 6                                                             \\
        Test11 & 29  & 19      & -                                                             \\
        Test12 & 2   & 2       & -                                                             \\
        Test13 & 2   & 1       & 1                                                             \\
        Test14 & 2   & 1       & 1                                                             \\
        \hline
    \end{tabular}



    \subsection{Analyse des tests}\label{Analyse des tests}
    \subsubsection{Optimisation}
    \par
    \indent
    Nous avons donc remarqué suite à cela que le test 4 prenait beaucoup de temps,
    et plus généralement les tests concernant la carte du monde (Test 3, 4, 9, 10)
    et les tests concernant la carte de la france (Test 5,6,11,12). Ces tests
    devraient être les plus rapides, car ils contiennent le moins de polygones, et
    les algorithmes sont quadratiques et linéaire. Nous avons compris que le
    problème venait du fait que les multi-polygones prenait énormément de place et nous avons refait les tests
    \par
    \subsubsection{Influence du N}
    \par
    \indent
    Nous remarquons aussi que le nombre N(nombre maximum d'enfants) n'a pas
    d'influence notable sur la rapidité de notre algorithme.
    \par
    \subsubsection{Algorithme linéaire vs Algorithme quadratique}
    \par
    \indent
    Nous avons aussi remarqué que l'algorithme linéaire prend beaucoup plus de temps
    à chercher un point dans un polygone. En moyenne, la recherche prend 39,9
    millisecondes pour l'algorithme linéaire et 13,25 millisecondes pour
    l'algorithme quadratique.
    \par
    \indent
    \par
    Nous avons donc comparé les temps d'exécution. En moyenne, le temps
    d'exécution sur 6 tests pour l'algorithme quadratique est de 6,357 secondes.
    Le temps d'exécution sur les mêmes tests pour l'algorithme linéaire est de
    3,087 secondes. La création de l'arbre est donc beaucoup plus rapide pour
    l'algorithme linéaire que pour le quadratique, ce qui est attendu vu la
    complexité de chaque algorithme. Cependant, il est intéressant de noter que la
    fonction de recherche prend beaucoup plus de temps dans l'algorithme linéaire.
    \par

    \section{Conclusion}
    \par
    \indent
    En conclusion, nous avons remarqué une différence notable de la fonction
    search entre l'algorithme linéaire et quadratique, l'algorithme quadratique
    étant beaucoup plus rapide. Cependant pour la création du R-Tree, l'algorithme
    linéaire est bien plus rapide. Nous avons également optimisé notre programme
    grâce à la décomposition de multi-polygones en polygones.
    \par
    \section{Annexes}
    \begin{figure}[h]
        \caption{Before optimization}\label{Figure 1}
        \includegraphics[width=\textwidth]{beforeopti.png}\label{fig:beforeopti}
    \end{figure}

    \begin{figure}[h]
        \caption{After optimization}\label{Figure 2}
        \includegraphics[width=\textwidth]{afterOpti.png}\label{fig:afteropti}
    \end{figure}

    \nocite{*}
    \bibliographystyle{apalike}
    \bibliography{myBib.bib}
    \addcontentsline{toc}{section}{\numberline{}References}

\end{large}

\end{document}

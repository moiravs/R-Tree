
\documentclass[utf8]{article}
\usepackage[utf8]{inputenc}

\usepackage[parfill]{parskip}
\usepackage{booktabs}
\usepackage{amsmath}
\usepackage{amssymb}
\usepackage{amsfonts}
\usepackage{graphicx}
\usepackage{float}
\usepackage{listingsutf8}
\usepackage{listings}
\usepackage{graphicx}
\usepackage{hyperref}
\usepackage{fullpage}
\usepackage{lipsum}

\nocite{*}
\setlength\parindent{24pt}

\begin{document}
\begin{titlepage}


\author{Andrius Ezerskis \& Moïra Vanderslagmolen}
\title{Projet d'Algorithmique: R-trees}

\maketitle
\end{titlepage}
\tableofcontents
\newpage
\begin{large}



\section{Introduction}
\indent
\par


\par

\section{Structure du code}

\par
\subsection{MBRNode}
MBRNode représente les noeuds des R-Tree. des polygones

\subsection{RTreeLinear}
\label{RTreeLinear}

RTreeLinear représente l'implémentation du R-Tree avec l'algorithme de split linéaire.

\subsection{RTreeQuadratic}
\label{RTreeQuadratic}

\subsection{RTree}

RTree est une classe abstraite. Elle permet de regrouper ensemble les méthodes
communes à \nameref{RTreeLinear} et \nameref{RTreeQuadratic}, par exemple la
recherche d'un noeud (searchNode), l'initialisation de la classe


\subsection{FileLoader}

Cette classe permet de charger le fichier en mémoire. Nous avons implémenté
cette classe afin de facilement changer la façon dont les fichiers sont chargés
en mémoire.

\section{Conclusion}
\indent
\par
\par

\section{Bibliographie}

\section{Annexes}

\end{large}

\end{document}


\documentclass[utf8]{article}
\usepackage[utf8]{inputenc}

\usepackage[parfill]{parskip}
\usepackage{booktabs}
\usepackage{amsmath}
\usepackage{amssymb}
\usepackage{amsfonts}
\usepackage{graphicx}
\usepackage{float}
\usepackage{listingsutf8}
\usepackage{listings}
\usepackage{graphicx}
\usepackage{hyperref}
\usepackage{fullpage}
\usepackage{lipsum}

\nocite{*}
\setlength\parindent{24pt}

\begin{document}
\begin{titlepage}


\author{Andrius Ezerskis \& Moïra Vanderslagmolen}
\title{Projet d'Algorithmique: R-trees}

\maketitle
\end{titlepage}
\tableofcontents
\newpage
\begin{large}



\section{Introduction}
\indent
\par


\par
\section{RTree}
\subsection{Création RTree}

\subsection{Split Linéaire}
Au début, nous copions l'entiereté du vecteur des enfants du node choisi pour
split. Puis nous vidons ce vecteur et nous rajoutons les seeds choisies. Et puis
seulement avec le vecteur copié nous ajoutions au fur et à mesure au seeds
choisies. Nous nous sommes dit qu'on perdait bcp de temps à copier l'entiereté
du vecteur, surtout sur des gros vecteurs.

Nous avons donc changé l'ordre, d'abord nous appelons pickNext pour qu'il
rajoute les enfants aux seeds, puis nous vidons le vecteur du node à spliter et
rajoutons les splits seeds.

Grâce à cette amélioration, les 8 tests que nous avons fait tourner ont pris
5,865 secondes au total au lieu de 6,778 secondes.
\subsection{Split Quadratique}


\section{Structure du code}

\par
\subsection{MBRNode}
MBRNode représente les noeuds des R-Tree. Elle contient un label, un polygone,
des enfants (sauf si c'est une feuille) et un parent (sauf si c'est la racine de
l'arbre)

\subsection{RTreeLinear}
\label{RTreeLinear}

RTreeLinear représente l'implémentation du R-Tree avec l'algorithme de split linéaire.
\subsection{RTreeQuadratic}
\label{RTreeQuadratic}

\subsection{RTree}

RTree est une classe abstraite. Elle permet de regrouper ensemble les méthodes
communes à \nameref{RTreeLinear} et \nameref{RTreeQuadratic}, par exemple la
recherche d'un noeud (searchNode), l'initialisation de la classe


\subsection{FileLoader}

Cette classe permet de charger le fichier en mémoire. Nous avons implémenté
cette classe afin de facilement changer la façon dont les fichiers sont chargés
en mémoire.

\section{Conclusion}
\indent
\par
\par

\section{Bibliographie}

\section{Annexes}

\end{large}

\end{document}
